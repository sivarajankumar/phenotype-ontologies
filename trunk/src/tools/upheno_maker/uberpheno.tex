
\subsection{Background}

\section{Methods}

\subsection{Ontologies Used}

\begin{enumerate}
\item mp
\item hp
\item mp-hp-equivs*
\item mp-equivalence-axioms
\item hp-equivalence-axioms
\item zpheno*
\item fma
\item zfa
\item ma
\item uberon
\item uberon-bridge-fma*
\item uberon-bridge-zma*
\item uberon-bridge-ma*
\item [TODO - we will add GO, CL in future]
\end{enumerate}

These are all combined into a single ontology called mp-hp-ext-merged-uberon. This is done using OWLTools, with the following command

\begin{verbatim}
owltools ONT1.owl ONT2.owl ... --merge-support-ontologies -o file://`pwd`/mp-hp-ext-merged-uberon.owl
\end{verbatim}

The files marked ``*'' are derived ontologies - generation of these files are described here

\subsection{Generation of MP-HP equivalencies}

An equivalence pair $(C1,C2)$ is generated if $C1$ and $C2$ are
reciprocal best matches according to synonyms.

\textbf{Synonym matching:} We use the SWI-Prolog porter_stem method in
the nlp library to tokenize and stem all tokens for each synonym,
omitting stemming for acronyms or special tokens (any token with >1
upper case character, or any non-alphanumeric characters). We filter
out prepositions and determinants. We apply special purpose
substitutions (e.g. changing I, first, 1st to 1), and use the Obol
library of biological relational adjectives to substitute adjectives
such as ``renal'' with their noun form (e.g. ``kidney'').

Finally, these normalized tokens are ordered alphabetically and
concatenated without whitespace to produce a normalized label.

\textbf{Reciprical best matches:} Any class pair matches if they share
the same normalized label. A class pair C1, C2 is a best match if
there is no C2' such that C1 matches C2', and the both synonyms used
for the latter had stronger scope than both the synonyms used for the
former. We treat exact synonyms and primary labels are equal, and as
being stronger than broad, narrow or related, which are weaker. We
only consider inter-ontology matches.

A best match C1,C2 is reciprocal if C2 has best match C1.

Each equivalence pair is then translated to OWL equivalence axioms:

\begin{verbatim}
EquivalentClasses(C1 C2)
\end{verbatim}

We then checked all equivalence pairs for structural difference across
the two ontologies, and resolved these [in progress].

The pipeline for making this is in src/ontology/hp-mp

[dir name may change]

\textbf{Filtering:} We maintain a manually edited file of explicitly
djsoint MP-HP class pairs. The purpose of this file is to exclude
false positive alignments based on non-exact synonyms. Any potential
equivalence pair C1-C2 is excluded if there is a disjoint pair D1-D2
and C1 and C2 are inferred subclasses of D1 and D2 respectively
(recall that subclass is a reflexive relation, so this includes the
case where C1=D1 and C2=D2).

\subsection{Generation of ZFIN phenotype ontology}

We downloaded pheno_obo from ZFIN. Annotations include 3 columns: E, Q
and E2. Currently the semantics of the E2 column are mixed - they
either represent the dependent entity (towards), or the entity to
which E is part of. We treat them all as being partOf.

The translation follows \cite{MungallPheno2009}, i.e.

\begin{verbatim}
E Q ==> Q and BFO_0000052 some E
E Q E2 ==> Q and BFO_0000052 some (E and BFO_0000050 some E2)
\end{verbatim}

Where BFO:0000052 is the hasPhenotype property, and BFO:0000052 is the
inheres in property.

For each distinct class expression, we declare a class using a
skolemized URI and an-autogenerated name. The skolemized URI is
essentially a concatenation of the IDs used in the class expression,
and the labels is a concatenization of their names.

\subsection{Uberpheno Construction}

\subsubsection{Ontology Merging}

The ontology merge process takes a collection of input phenotype
ontologies $P$ (by default: \{mp, hp, zpheno\}) together with their
associated equivalence axioms, and produces a new ontology $O$
(i.e. uberpheno).

\textbf{initialize}. First of all, we initialize $O$ to be an empty
ontology with no axioms or declarations.

\textbf{Definition: representative members of equivalence sets}. We take all
EquivalentClasses axioms connecting two named classes in $P$, and
build a table of class pairs (recall from the above that this is
derived by text matching). For each equivalence set, we designate one
member as the \emph{representative}, and the others as
\emph{secondary}. The criteria is arbitrary, based on alphabetic
ordering of the IRI. $A$ has representative $B$ if A is equivalent
to B, and the IRI for A alphabetically precedes B, and there is no
equivalent of B such that B precedes this class.

For example, given:

\begin{verbatim}
EquivalentClasses(HP_1 MP_1)
EquivalentClasses(HP_2 MP_2)
EquivalentClasses(HP_3 MP_3)
\end{verbatim}

The class pair HP1-MP1 has representative HP1.

The representative classes are used when we create the unified
phenotype ontology. We will merge class pairs into a new class [todo:
this could be simplified by eliminating the notion of a representative]

\textbf{Definition: Phenotype IRI mapping}. We define a mapping
function $u$ between named classes in $P$ and named classes in $O$
(i.e. between species-centric phenotype classes and species-generic
phenotype classes). The mapping function takes the IRI of a phenotype
class in $P$ and rewrites it as a uberpheno IRI. Specifically, we
tokenize on the underscore character and then append the resulting
tokens onto the prefix obo:UPHENO_ base.

For example:

$u$(obo:HP\_0000001) =  obo:UPHENO\_HP0000001

\textbf{Merging into unified phenotype ontology}. We iterate
through all classes in $P$. We ignore all classes that have an
owl:deprecated annotation (i.e. are obsolete in the corresponding obo
file). We also ignore classes that have a representative (these are
handled separately, see below).

We merge each remaining phenotype class $C$ in $P$ into $O$, as follows.

First we create a class declaration for $u(C)$ in $O$. For example, if
P contains a class declaration obo:HP\_0000001, we create a class
declaration in O for obo:UPHENO\_HP0000001.

Then we copy and rewrite selected axioms across from $C$ to $u(C)$:

\begin{description}
\item all $SubClassOf$ axioms, where the first argument is $C$
\item all $EquivalentClasses$ axioms, where any argument is $C$
\item all $AnnotationAssertion$ or $PropertyAssertion$ axioms, where the second argument is $C$
\end{description}

The core phenotype ontologies typically include only subclass axioms
and annotation assertions. The equivalence axioms come from either the
logical definition files, or the automated text mappings.

We then repeat this for every $C'$ where $C'$ is equivalent to $C$ -
note that C' will be the representative of C.

For each axiom, we recursively traverse all class expressions, replace
IRIs $X$ in $P$ with $u(X)$. If $X$ has an uberon mapping,
then we replace $X$ with the corresponding uberon class. Finally,
if $X$ has a representative equivalent, then we replace with this.

For example,

\begin{verbatim}
subClassOf(MP_0000001 MP_0000002)
\end{verbatim}

becomes

\begin{verbatim}
subClassOf(UPHENO_MP0000001 UPHENO_MP0000002)
\end{verbatim}

At this point we save $O$ as uberpheno-asserted.owl

This ontology was manually inspected im Protege4. We also converted
obo using obolib and browsed in OBOEdit. Browsing in P4 is better
able to show equivalence axios.

We did not do reasoning in P4 or OE at this stage, as the lack of
axioms would deliver limited results. This is done after the next
stage, bringing in axioms from external ontologies.

\subsubsection{MIREOT}

We apply a MIREOT-like procedure to bring in a minimal set of axioms
required for supporting inferences on $O$.

We define a connectivity predicate $c$ as:

$$
c(x,y) \larr SubClassOf(x y)
$$

$$
c(x,y) \larr EquivalentClasses(x y)
$$

$$
c(y AND ...,y)
$$

$$
c(x SOME R y,y) \larr MireotProperty(p)
$$

Then set of properties is configurable, but by default is part of,
inheres in, inheres in part of, and qualifier. This is similar to a
restricted version of the syntactic locality module extractor in the
OWLAPI.

We calculate the closure of $c$ using depth first search. For each
object visisted in the search, we import all SubClassOf,
EquivalentClasses and AnnotationAssertion axioms.

We save $O$ as uberpheno-asserted-full. This can be viewed in an
ontology editor. Reasoning can be done in P4 or OE, the latter using
the Rule Based Reasoner. Use of P4 is recommended.

\subsubsection{Reasoning}

We feed the results of $$O$$ into pellet[todo - jcel] and initialize reasoning. We
expect that $$O$$ will fall into the EL class, but apply a procedure
filtering all non-EL axioms in order to guarantee that the more
scalable EL algorithm is selected.

The resulting subclass and equivalence axioms are added to $O$.

We save $O$ as uberpheno-inferred.

We manually checked uberpheno-inferred for incorrect inferences [todo]
- for any we found, we loaded uberpheno-asserted-full into P4, ran the
reasoner, and use explanation capabilities.

\subsubsection{Optional extras}

The above uses very conservative transformations to avoid producing
weird inferences associated with uberpheno 0.0

We will gradually introduce additional axioms here (for example,
creating new LCSs). TODO

* selected transformation of patterns (e.g has part)
* ...








